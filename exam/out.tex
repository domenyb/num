\documentclass{article}
\usepackage{graphicx}

\begin{document}
\title{Generalized eigenvalue problem}
\author{Balazs Doemeny \\ Exam project}
\date{}
\maketitle

\begin{abstract}
\centering
Part of the exam project
\end{abstract}


The exam project part 1) did involve the task to prove analytically that the original $Ax=\lambda Nx$ can be represented as $By=\lambda y$, where $B=  \sqrt{D}^{-1} V^T A V \sqrt{D}^{-1} $ and $ y=\sqrt{D} V^T x $
\\
For this, first transform $N=VDV^T$ and multiply both sides by $\sqrt{D}^{-1} V^T$ from the left-hand side. This gives (after using that $V^TV= identity matrix$ and moving the scalar value $\lambda$)
$$\sqrt{D}^{-1} V^T A x =\lambda \sqrt{D} V^T x $$
On the right-hand side, we already do have the outcome we wanted, $\lambda \sqrt{D} V^T x = \lambda y$
\\
Taking a closer look at the left-hand side it is $$ \sqrt{D}^{-1} V^T A I x $$ Since I can freely introduce indentity matrices into the equation, that doesn't have any effect on it.
\\
use $V^TV=VV^T= identity matrix$ again, and introduce another I:  $$ \sqrt{D}^{-1} V^T A V I V^T x $$
use $\sqrt{D}^{-1} \sqrt{D} = I$
$$ \sqrt{D}^{-1} V^T A V \sqrt{D}^{-1} \sqrt{D} V^T x $$
This is exaclty $By$, so we did prove the equivalence of the two expressions.


\end{document}
